
\usepackage{tikz}

%
% Процедура-склонялка :)
% полезная штука для склонения "риунков", "страниц" и
% прочего подобного добра в курсачах
%
%  Синтаксис:
% \pluralize{num}{начало слова}{ок.0}{ок.1}{ок.2..4}{ок.5итд}
%
% (ок. == окончание.)
%
%  Пример:
%  "10 \pluralize{10}{рисун}{ков}{ок}{ка}{ков}"
%  даст:
%  "10 рисунков"
%
%

\newcommand{\pluralizeothers}[6]{%
\expandafter\ifcase#1%
#2#3\or% 0
#2#4\or% 1
#2#5\or% 2
#2#5\or% 3
#2#5\or% 4
#2#6\or% 5
#2#6\or% 6
#2#6\or% 7
#2#6\or% 8
#2#6\else% 9
#2#6\fi\relax% всё остальное...
}

\newcommand{\pluralize}[6]{%
\pgfmathtruncatemacro{\hundred}{mod(int(#1),100)}%
\pgfmathtruncatemacro{\decade}{mod(\hundred,10)}%
\pgfmathparse{and(\hundred>9,\hundred<20) ? "#2#6" : "\pluralizeothers{\decade}{#2}{#3}{#4}{#5}{#6}"}\pgfmathresult%
}