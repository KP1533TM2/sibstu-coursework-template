
% Две эти переменные нужны для "калибровки" положения рамок на листе.
% Со стандартными полными размерами рамки ГОСТ у некоторых принтеров
% большие проблемы. Например, у моего престарелого HP LaserJet 6L.
\def\HorizontalFrameOffset{15.3}
\def\VerticalFrameOffset{-1.5}

\usepackage{chngcntr}
\counterwithout{figure}{chapter}
\usepackage{totcount}
\regtotcounter{figure}
\usepackage{wrapfig}

\usepackage{tocloft}
\tocloftpagestyle{framedtoc}

\usepackage{lastpage}
\usepackage{titlesec}

\usepackage{geometry}
\textwidth=165mm
\textheight=250mm
\oddsidemargin=10mm
\evensidemargin=0mm
\headsep=10mm
\topmargin=-25mm
\unitlength=1mm


\def\VL{\line(0,1){15}}
\def\HL{\line(1,0){185}}
\def\Box#1#2{\makebox(#1,5){#2}}

\def\frameTitle
{\sl\small\noindent\hbox to 0pt
	{\vbox to 0pt
		{
			\noindent
			\begin{picture}(185,287)(\HorizontalFrameOffset,\VerticalFrameOffset)
				\linethickness{0.5mm}
				\put(0,0){\framebox(185,285){}}
			\end{picture}
		}
	}
}

\def\frameNormal
{\sl\small\noindent\hbox to 0pt
	{\vbox to 0pt
		{
			\noindent
			\begin{picture}(185,287)(\HorizontalFrameOffset,\VerticalFrameOffset)
				\linethickness{0.5mm}
				\put(0,0){\framebox(185,285){}}
				\put(0, 15)\HL
				\multiput(0, 5)(0, 5){2}{\line(1,0){65}}
				\put(65, 0){\VL\makebox(110,15){\large\sc\rightmark}}
				
				\put(175,10){\line(1,0){10}}
				
				{
					\fontsize{10pt}{13pt}\normalfont\selectfont
					\put(7,0){\VL\Box{10}{Лит.}}
					\put(17, 0){\VL\Box{23}{\No~докум.}}
					\put(40, 0){\VL\Box{15}{Подп.}}
					\put(55, 0){\VL\Box{10}{Дата}}
					\put(175,10){\Box{10}{Лист}}
					\put(175, 0){\VL\makebox(10,10){\normalsize\thepage}}
					\fontsize{14pt}{18.2pt}\selectfont
					\put(65,5){\Box{120}{\docCode}}
				}
				% всё, что не влезло в 10 пунктов, здесь:
				{
					\fontsize{8pt}{13pt}\normalfont\selectfont
					\put(0, 0){\Box{7}{Изм.}}
				}
			\end{picture}
		}
	}
}

\def\frameTOC
{\sl\small\noindent\hbox to 0pt
	{\vbox to 0pt
		{
			\noindent
			\begin{picture}(185,287)(\HorizontalFrameOffset,\VerticalFrameOffset)
				\linethickness{0.5mm}
				\put(0,0){\framebox(185,285){}}
				\put(0, 40)\HL
				\multiput(0, 5)(0, 5){7}{\line(1,0){65}}
				\put(7, 25){\line(0,1){15}}
				\put(17, 0){\line(0,1){40}}
				\put(40, 0){\line(0,1){40}}
				\put(55, 0){\line(0,1){40}}
				\put(65, 0){\line(0,1){40}}
				\put(65, 25){\line(1,0){120}}
				\put(135, 0){\line(0,1){25}}
				\put(135, 15){\line(1,0){50}}
				\put(135, 20){\line(1,0){50}}
				\put(140, 15){\line(0,1){5}}
				\put(145, 15){\line(0,1){5}}
				\put(150, 15){\line(0,1){10}}
				\put(165, 15){\line(0,1){10}}
				{
					\fontsize{10pt}{13pt}\normalfont\selectfont
					\put(7, 25){\Box{10}{Лист}}
					\put(17, 25){\Box{23}{\No~докум.}}
					\put(40, 25){\Box{15}{Подп.}}
					\put(55, 25){\Box{10}{Дата}}
					\put(0,0){\Box{17}{Утв.}}
					\put(0,5){\Box{17}{Н.контр.}}
					\put(0,15){\Box{17}{Пров.}}
					\put(0,20){\Box{17}{Разраб.}}
					\put(135,20){\Box{15}{Лит.}}
					\put(150,20){\Box{15}{Лист}}
					\put(165,20){\Box{20}{Листов}}
					\put(150,15){\Box{15}{\thepage}}
					\put(165,15){\Box{20}{\pageref{LastPage}}}
					\fontsize{14pt}{18.2pt}\selectfont
					%\put(65,10){\makebox(70,20){\docTitle}}
					\put(65,0){\makebox(70,25){\parbox{66mm}{\centering\docTitle}}}
					\put(65,30){\Box{120}{\docCode}}
					\put(135,5){\Box{50}{\docAuthorBox}}
				}
		
				% всё, что не влезло в 10 пунктов, здесь:
				{
					\fontsize{8pt}{13pt}\normalfont\selectfont
					\put(0, 25){\Box{7}{Изм.}}
					\put(135,15){\Box{15}{Кр}}
					\put(17,15){\Box{23}{\docDirector}}
					\put(17,20){\Box{23}{\docAuthorName}}
				}
			\end{picture}
		}
	}
}

\makeatletter
\def\ps@framedcontent{
\renewcommand{\@oddhead}{\frameNormal}
\renewcommand{\@evenhead}{\frameNormal}
\def\@oddfoot{}
\def\@evenfoot{}
}
\def\ps@framedtoc{
\renewcommand{\@oddhead}{\frameTOC}
\renewcommand{\@evenhead}{\frameTOC}
\def\@oddfoot{}
\def\@evenfoot{}
}

\def\ps@framedtitlepage{
\renewcommand{\@oddhead}{\frameTitle}
\renewcommand{\@evenhead}{\frameTitle}
\def\@oddfoot{}
\def\@evenfoot{}
}
\makeatother


\makeatletter
\renewcommand\chapter{\par%
  \thispagestyle{plain}%
  \global\@topnum\z@
  \@afterindentfalse
  \secdef\@chapter\@schapter}
\makeatother


\newcommand{\rubberyField}[2]{\hspace*{#1}\hbox to 0cm{\raisebox{-1em}{\footnotesize#2}}\hspace*{-#1}\hrulefill}