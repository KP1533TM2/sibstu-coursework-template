%
% Это введение.
%
%
%
{

\pagestyle{framedcontent}
\pagebreak
\cwNchapter{Введение}
С появлением в начале 1970-х годов первых программируемыx
постоянныx запоминающиx устройств (\engl{PROM, programmable read only
memory}), в истории микроэлектроники имеется тенденция к
развитию устройств с программируемой логикой как вычислительных
устройств, используемых для решения широкого круга задач цифровой
обработки информации.

В то же время в составе элементной базы цифровой схемотехники
отмечается переход от интегральных микросхем (ИМС) малой и средней
степени интеграции к большим (БИС, LSI) и сверхбольшим (СБИС, VLSI)
интегральным микросхемам.

Особенно сильное влияние на развитие цифровых вычислительных
устройств оказало создание первых микропроцессоров, что привело
к широкому внедрению цифровых технологий обработки информации.
Однако микропроцессоры не всегда приемлимы при решении задач в
цифровой схемотехники: работа микропроцессора основана на микропрограмме
и состоит из последовательности шагов конечной длительности, в то время
как для многих задач (в том числе, связанных и с обеспечением работы
самих МП) требуются устройства с минимальной задержкой выполнения
логических функций, что может быть обеспечено одним из трёх способов:

\begin{itemize}
\item Использованием наборов стандартной дискретной цифровой логи-
ки общего применения, например, наборов логических микросхем
74-й или 4000-й серии (TTL, CMOS) и типовых периферийных БИС из микропроцессорного комплекта;
\item Использованием заказных ИС (так называемых ASIC, \engl{appication-specific integrated circuit});
\item Использованием программируемых логических интегральных схем (PLD - Programmable Logic Devices).
\end{itemize}

\pagebreak
}