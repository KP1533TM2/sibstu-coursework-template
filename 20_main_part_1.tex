{
\pagestyle{framedcontent}
\cwchapter{Раздел 1}
Смотрите в файлик \texttt{20\_main\_part\_1.tex} для примеров добавления картинок,
списков, и всего подобного.

%-------------------------------------------------------------------------------
% Пример определения картинки слева от текста
%-------------------------------------------------------------------------------
%\begin{wrapfigure}[12]{r}{0.34\linewidth}
%\includegraphics[scale=0.35]{images/7400.jpg}
%\caption{Микросхема дискретной логики 4И (7400).}
%\label{ic-7400}
%\end{wrapfigure}

%-------------------------------------------------------------------------------
% Пример определения точечного списка по левому краю
%-------------------------------------------------------------------------------
%\begin{flushleft}
%\begin{itemize}
%\item Базовые логические элементы (NOT, AND, OR);
%\item Более сложные логические элементы (полусумматор, XOR);
%\item Память (триггеры, регистры и их наборы, RAM, PROM);
%\item Счетчики, компараторы;
%\item Мультиплексоры, шифраторы, дешифраторы;
%\end{itemize}
%\end{flushleft}

%-------------------------------------------------------------------------------
% Пример определения картинки сначала листа
%-------------------------------------------------------------------------------
%\begin{figure}[htp]
%\centering
%\includegraphics[scale=0.30]{images/TRS80M100_inside.jpg}
%\caption{Материнская плата Model 100.}
%\label{trs80-mobo}
%\end{figure}

%-------------------------------------------------------------------------------
% Пример определения кода Verilog
%-------------------------------------------------------------------------------
%\begin{minipage}{\linewidth}
%Пример кода Verilog:
%\lstinputlisting[language=Verilog]{code_examples/verilog_count.v}
%\end{minipage}

\cwsection{Секция 1}
lorem ipsum
\cwsection{Секция 2}
ipsum lorem
\cwsection{Секция 3}
to be or not to be, that is the question

\newpage
}