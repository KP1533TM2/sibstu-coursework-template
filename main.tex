%
% Рекомендуется использовать редактор Texmaker, с этим (main.tex) файлом, как основным.
%
%  TODO!!!
%  Сделать подписи к таблицам, если таковые имеются
%

\documentclass{sibstu}
\usepackage{textcase} 
\usepackage{setspace}
\usepackage{graphicx}
\usepackage{listingsutf8}
\usepackage{refcount}
\usepackage{lastpage}
\usepackage{enumitem}
\usepackage{fancyvrb}
\fvset{fontsize=\footnotesize,frame=single}
\setlist[itemize]{noitemsep}
\setlist{nolistsep}
\usepackage[figure,table]{totalcount}
\lstset{columns=fixed,language=VHDL,basicstyle=\footnotesize,breaklines=true,inputencoding=utf8/cp1251}

% Раскомменьте это, если вам нужны листинги языка AHDL
%\lstdefinelanguage{AHDL}%
  {morekeywords={AND,BEGIN,BIDIR,BITS,BURIED,CARRY,CASCADE,CASE,CLIQUE,CONNECTED_PINS,CONSTANT,
DEFAULTS,DESIGN,DEVICE,DFF,DFFE,ELSE,ELSIF,END,EXP,FUNCTION,GLOBAL,GND,IF,INCLUDE,INPUT,IS,JKFF,JKFFE,
LATCH,LCELL,MACHINE,MACRO,MCELL,NAND,NODE,NOR,NOT,OF,OPTIONS,OR,OTHERS,OUTPUT,RETURNS,SRFF,SRFFE,
SOFT,STATES,SUBDESIGN,TABLE,TFF,TFFE,THEN,TITLE,TRI,VARIABLE,VCC,WHEN,WITH,X,XNOR,XOR},%
 sensitive=f,%
 morecomment=[l]--,%
 morecomment=[s]{\%}{\%},%
 morestring=[d]{"}%
}[keywords,comments,strings]%

% Строка министертсва
\def\ministryTitle{Министерство образования и науки РФ}

% Строка статуса учрежденя
\def\institutionStatus{Федеральное государственное бюджетное образовательное учреждение высшего профессионального образования}

% Строка названия учреждения (регистр не важен, строка автоматически пишется капсом там где нужно)
\def\institutionName{Сибирский Государственный Технологический Университет}

% Строка названия кафедры
\def\departmentName{Кафедра системотехники}

% Заголовок работы
\def\docTitle{Программируемые логические интегральные схемы}

% Тип работы
\def\docStatus{Пояснительная записка}

% Группа
\def\groupNum{26-6}

% Руководитель
\def\docDirector{Коляда А.В.}

% Название дисциплины
\def\subjectName{Вычислительные системы}

% Факультет
\def\faculty{Автоматизации и информационных технологий, гр. \groupNum}

% Разработал
\def\docAuthorStatus{Студент группы {\groupNum}}
\def\docAuthorName{Гордеев А.Е.}
\def\docAuthorExactName{Алексей}
\def\docAuthorExactSurname{Гордеев}
\def\docAuthorExactFathername{Евгеньевич}

% Эта строка появляется в коробочке в правом нижнем углу на листе оглавления
\def\docAuthorBox{СибГТУ, гр. 26-6}

% Шифр
\def\docCode{СТ.000000.108 ПЗ}

% Город
\def\docCity{Красноярск}

% Год
\def\docYear{2013}

%
% Две эти переменные нужны для "калибровки" положения рамок на листе.
% Со стандартными полными размерами рамки ГОСТ у некоторых принтеров
% большие проблемы. Например, у моего престарелого HP LaserJet 6L.
\def\HorizontalFrameOffset{15.3}
\def\VerticalFrameOffset{-1.5}
\def\FrameWidth{185}
\def\FrameHeight{285}

\textwidth=165mm
\textheight=250mm
\oddsidemargin=10mm
\evensidemargin=0mm
\headsep=10mm
\topmargin=-25mm
\unitlength=1mm

%---------------------------------------------------------------------------------------------------
%  РАМКИ
%---------------------------------------------------------------------------------------------------
%---------------------------------------------------------------------------------------------------

\def\VL{\line(0,1){15}}
\def\HL{\line(1,0){185}}
\def\Box#1#2{\makebox(#1,5){#2}}

\def\frameTitle
{\sl\small\noindent\hbox to 0pt
	{\vbox to 0pt
		{
			\noindent
			\begin{picture}(185,287)(\HorizontalFrameOffset,\VerticalFrameOffset)
				\linethickness{0.5mm}
				\put(0,0){\framebox(\FrameWidth,\FrameHeight){}}
			\end{picture}
		}
	}
}

\def\frameNormal
{\sl\small\noindent\hbox to 0pt
	{\vbox to 0pt
		{
			\noindent
			\begin{picture}(185,287)(\HorizontalFrameOffset,\VerticalFrameOffset)
				\linethickness{0.5mm}
				\put(0,0){\framebox(\FrameWidth,\FrameHeight){}}
				\put(0, 15)\HL
				\multiput(0, 5)(0, 5){2}{\line(1,0){65}}
				\put(65, 0){\VL\makebox(110,15){\large\sc\rightmark}}
				
				\put(175,10){\line(1,0){10}}
				
				{
					\fontsize{10pt}{13pt}\normalfont\selectfont
					\put(7,0){\VL\Box{10}{Лит.}}
					\put(17, 0){\VL\Box{23}{\No~докум.}}
					\put(40, 0){\VL\Box{15}{Подп.}}
					\put(55, 0){\VL\Box{10}{Дата}}
					\put(175,10){\Box{10}{Лист}}
					\put(175, 0){\VL\makebox(10,10){\normalsize\thepage}}
					\fontsize{14pt}{18.2pt}\selectfont
					\put(65,5){\Box{120}{\docCode}}
				}
				% всё, что не влезло в 10 пунктов, здесь:
				{
					\fontsize{8pt}{13pt}\normalfont\selectfont
					\put(0, 0){\Box{7}{Изм.}}
				}
			\end{picture}
		}
	}
}

\def\frameTOC
{\sl\small\noindent\hbox to 0pt
	{\vbox to 0pt
		{
			\noindent
			\begin{picture}(185,287)(\HorizontalFrameOffset,\VerticalFrameOffset)
				\linethickness{0.5mm}
				\put(0,0){\framebox(\FrameWidth,\FrameHeight){}}
				\put(0, 40)\HL
				\multiput(0, 5)(0, 5){7}{\line(1,0){65}}
				\put(7, 25){\line(0,1){15}}
				\put(17, 0){\line(0,1){40}}
				\put(40, 0){\line(0,1){40}}
				\put(55, 0){\line(0,1){40}}
				\put(65, 0){\line(0,1){40}}
				\put(65, 25){\line(1,0){120}}
				\put(135, 0){\line(0,1){25}}
				\put(135, 15){\line(1,0){50}}
				\put(135, 20){\line(1,0){50}}
				\put(140, 15){\line(0,1){5}}
				\put(145, 15){\line(0,1){5}}
				\put(150, 15){\line(0,1){10}}
				\put(165, 15){\line(0,1){10}}
				{
					\fontsize{10pt}{13pt}\normalfont\selectfont
					\put(7, 25){\Box{10}{Лист}}
					\put(17, 25){\Box{23}{\No~докум.}}
					\put(40, 25){\Box{15}{Подп.}}
					\put(55, 25){\Box{10}{Дата}}
					\put(0,0){\Box{17}{Утв.}}
					\put(0,5){\Box{17}{Н.контр.}}
					\put(0,15){\Box{17}{Пров.}}
					\put(0,20){\Box{17}{Разраб.}}
					\put(135,20){\Box{15}{Лит.}}
					\put(150,20){\Box{15}{Лист}}
					\put(165,20){\Box{20}{Листов}}
					\put(150,15){\Box{15}{\thepage}}
					\put(165,15){\Box{20}{\pageref{LastPage}}}
					\fontsize{14pt}{18.2pt}\selectfont
					%\put(65,10){\makebox(70,20){\docTitle}}
					\put(65,0){\makebox(70,25){\parbox{66mm}{\centering\docTitle}}}
					\put(65,30){\Box{120}{\docCode}}
					\put(135,5){\Box{50}{\docAuthorBox}}
				}
		
				% всё, что не влезло в 10 пунктов, здесь:
				{
					\fontsize{8pt}{13pt}\normalfont\selectfont
					\put(0, 25){\Box{7}{Изм.}}
					\put(135,15){\Box{15}{Кр}}
					\put(17,15){\Box{23}{\docDirector}}
					\put(17,20){\Box{23}{\docAuthorName}}
				}
			\end{picture}
		}
	}
}

\makeatletter
\def\ps@framedcontent{
\renewcommand{\@oddhead}{\frameNormal}
\renewcommand{\@evenhead}{\frameNormal}
\def\@oddfoot{}
\def\@evenfoot{}
}
\def\ps@framedtoc{
\renewcommand{\@oddhead}{\frameTOC}
\renewcommand{\@evenhead}{\frameTOC}
\def\@oddfoot{}
\def\@evenfoot{}
}

\def\ps@framedtitlepage{
\renewcommand{\@oddhead}{\frameTitle}
\renewcommand{\@evenhead}{\frameTitle}
\def\@oddfoot{}
\def\@evenfoot{}
}
\makeatother


\makeatletter
\renewcommand\chapter{\par%
  \thispagestyle{plain}%
  \global\@topnum\z@
  \@afterindentfalse
  \secdef\@chapter\@schapter}
\makeatother


\newcommand{\rubberyField}[2]{\hspace*{#1}\hbox to 0cm{\raisebox{-1em}{\footnotesize#2}}\hspace*{-#1}\hrulefill}


%\newcommand{\rubberyField}[2]{\hspace*{#1}\hbox to 0cm{\raisebox{-1em}{\footnotesize#2}}\hspace*{-#1}\hrulefill}
%
%  Форматирование заголовков
%

%\newcommand{\addURL}[4]{{#1} [Электронный ресурс]: электронная система: база данных содержит электронные статьи / {#2}, {#3}. -- Режим доступа: {#4}}

\newcommand{\addURL}[4]{Электронный ресурс {#4}}

\newcommand{\addBook}[6]{{#1} {#2} -- {#3}: {#4}, {#5}. -- {#6} c.}

% ACHTUNG, ACHTUNG! Юзаем \cwchapter и \cwsection для правильного
% оформления и автоматических ништяков :-)

\newcommand\cwchapter[1]{\chapter{\MakeUppercase{#1}}}
\newcommand\cwNchapter[1]{\chapter*{\MakeUppercase{#1}}\addcontentsline{toc}{chapter}{\MakeUppercase{#1}}}
\newcommand\cwNNchapter[1]{\chapter*{\MakeUppercase{#1}}}
\newcommand\cwsection[1]{\section{\MakeUppercase{#1}}}
\newcommand\cwNsection[1]{\section*{\MakeUppercase{#1}}\addcontentsline{toc}{section}{\MakeUppercase{#1}}}

% Эти строки определяют вид заголовков разделов и секций
% MakeUppercase здесь почему-то не работает (не разбирался), так что
% обойдемся костылями B-)
\titleformat{\chapter}[hang]{\fontsize{16pt}{13pt}\selectfont}{\thechapter\quad}{-6pt}{}
\titleformat{\section}[hang]{\fontsize{16pt}{13pt}\selectfont}{\thesection\quad}{-6pt}{}
\titleclass{\chapter}{straight}
\titleclass{\section}{straight}

% Отступы от заголовков
\titlespacing{\chapter}{0mm}{1.2em}{1.2em}
\titlespacing{\section}{0mm}{1.2em}{1.2em}

% Отступ красной строки
\setlength{\parindent}{1cm}

% Промежутки между параграфами
\setlength{\parskip}{.5\baselineskip}

% Эти строки описывают подписи к картинкам 
\usepackage{caption}
\captionsetup[figure]{name=Рисунок, justification=centering,labelsep=endash}

\usepackage{tikz}

%
% Процедура-склонялка :)
% полезная штука для склонения "риунков", "страниц" и
% прочего подобного добра в курсачах
%
%  Синтаксис:
% \pluralize{num}{начало слова}{ок.0}{ок.1}{ок.2..4}{ок.5итд}
%
% (ок. == окончание.)
%
%  Пример:
%  "10 \pluralize{10}{рисун}{ков}{ок}{ка}{ков}"
%  даст:
%  "10 рисунков"
%
%

\newcommand{\pluralizeothers}[6]{%
\expandafter\ifcase#1%
#2#3\or% 0
#2#4\or% 1
#2#5\or% 2
#2#5\or% 3
#2#5\or% 4
#2#6\or% 5
#2#6\or% 6
#2#6\or% 7
#2#6\or% 8
#2#6\else% 9
#2#6\fi\relax% всё остальное...
}

\newcommand{\pluralize}[6]{%
\pgfmathtruncatemacro{\hundred}{mod(int(#1),100)}%
\pgfmathtruncatemacro{\decade}{mod(\hundred,10)}%
\pgfmathparse{and(\hundred>9,\hundred<20) ? "#2#6" : "\pluralizeothers{\decade}{#2}{#3}{#4}{#5}{#6}"}\pgfmathresult%
}

% Нужно для последних версий babel, ибо из них убрана команда \No
\newcommand{\No}{\textnumero}

% Команда для англицизмов
\newcommand\engl[1]{англ. \emph{#1}}

\sloppy
%\renewcommand{\ttdefault}{cmtt} % Моноширинный шрифт
\begin{document}
{
% Здешние значения взяты из titleinfo.tex

{

\thispagestyle{framedtitlepage}
\enlargethispage{1.5cm}

\begin{center}
\ministryTitle\\
\institutionStatus\\
\MakeTextUppercase{\institutionName}\\
\departmentName

\vspace*{\fill}
\begin{center}
{\textbf{\fontsize{20pt}{26pt}\selectfont\docTitle}}\\
\vspace{\baselineskip}
\docStatus\\
(\docCode)
\end{center}
\vspace*{\fill}

\hfill\parbox{8cm}{
Руководитель:\\
\rubberyField{1cm}{(подпись)}\enspace\docDirector\\\\
\rubberyField{2cm}{(оценка, дата)}\\\\
\\
Разработал:\\
\docAuthorStatus\\
\rubberyField{1cm}{(подпись)}\enspace\docAuthorName\\\\
\rubberyField{2cm}{(дата сдачи)}\\\\
}

% Раскомментировать если потребуется
%\docCity, \docYear

\end{center}

}%				автоматический файл
{
\thispagestyle{empty}
\begin{center}
\MakeTextUppercase{\institutionName}

\begin{spacing}{1.3}
\textbf{\fontsize{16pt}{18pt}\selectfont\MakeTextUppercase{Задание на расчетно-графическую работу по архитектуре вычислительных систем}}
\end{spacing}
\end{center}

\noindent
{\textbf{Студент}\quad{{\docAuthorExactSurname} {\docAuthorExactName} {\docAuthorExactFathername}}}\\
{\textbf{Факультет}\quad{\faculty}}\\
{\textbf{Тема курсовой работы:}\quad{\docTitle}}\\

\textbf{Задания}
\begin{enumerate}
\item Представить информацию по архитектуре ПЛИС
\item Представить информацию по истории ПЛИС
\item Представить информацию по языкам описания аппаратного обеспечения (HDL)
\item Подвести итоги выполненной работы
\end{enumerate}
\vfill
\hfill\parbox{8cm}{
\textbf{\noindent{Задание выдано \hrulefill}}

{\noindent{\textbf{Руководитель} \docDirector}}
}
\newpage
}%			!	для заполнения			!
{
\thispagestyle{empty}
\cwNNchapter{Реферат}
% ВНИМАНИЕ! Необходимо править окончания слов вручную в зависимости от количества рисунков/страниц/источников!
% я пока не нашел пути автоматизировать это.
Пояснительная записка включает в себя \pageref{LastPage} страницы текста, \ref{literarure:num} использованных литературных источников, \total{figure} рисунков. 

Цель работы -- to be filled.

Ключевые слова: to be filled.

\newpage
}%		!	для заполнения			!
%
%	Этот файл, как правило, не нужно трогать, ибо он полностью автоматизирован
%
%
%
{
\thispagestyle{framedtoc}
\enlargethispage{-1.7cm}

\setlength{\cftbeforetoctitleskip}{-2em}
\setlength{\cftaftertoctitleskip}{0.1pt}

% Интервал элементов оглавления (все приравнены)
\setlength{\cftbeforechapskip}{0em}
\setlength{\cftbeforesecskip}{0em}
\setlength{\cftbeforepartskip}{0em}

% Отступ элементов оглавления (все приравнены)
\setlength{\cftchapindent}{0pt}
\setlength{\cftsecindent}{\cftchapindent}

%\renewcommand\cftsecafterpnum{\vskip\baselineskip}
%\renewcommand\cftchapafterpnum{\vskip\baselineskip}

\renewcommand\cftchapfont{\fontsize{16pt}{13pt}\selectfont}
\renewcommand\cftsecfont{\cftchapfont}
\renewcommand\cftchappagefont{\cftchapfont}
\renewcommand\cftsecpagefont{\cftchapfont}

\renewcommand\contentsname{\fontsize{16pt}{13pt}\selectfont\begin{center}\textbf{СОДЕРЖАНИЕ}\end{center}}
\renewcommand{\cftdot}{.}
\renewcommand{\cftdotsep}{0.5}
\renewcommand{\cftpartleader}{\cftdotfill{\cftdotsep}} % for parts
\renewcommand{\cftchapleader}{\cftdotfill{\cftdotsep}} % for chapters
%\renewcommand{\cftsecleader}{\cftdotfill{\cftdotsep}} % for sections, if you really want! (It is default in report and book class (So you may not need it).
% ----------------------------------------------------------------
% aasdfsadfasdfasdfasdf
% \lineskiplimit, установленный на отрицательное значение, заставляет TeX думать,
% что строчки никогда не накладываются (кончики буковок с одной строки не наклады-
% ваются на другую строку). Иначе в оглавлении строки будут незначительно, но таки
% заметно поскакивать, и становится некрасиво =\
% А spacing, собственно, задает интервал между строками, уже жестко.
%
\begin{spacing}{1.3}
\lineskiplimit=-\maxdimen\relax\tableofcontents
\end{spacing}

}%				автоматический файл
%
% Это введение.
%
%
%
{

\pagestyle{framedcontent}
\pagebreak
\cwNchapter{Введение}
С появлением в начале 1970-х годов первых программируемыx
постоянныx запоминающиx устройств (\engl{PROM, programmable read only
memory}), в истории микроэлектроники имеется тенденция к
развитию устройств с программируемой логикой как вычислительных
устройств, используемых для решения широкого круга задач цифровой
обработки информации.

В то же время в составе элементной базы цифровой схемотехники
отмечается переход от интегральных микросхем (ИМС) малой и средней
степени интеграции к большим (БИС, LSI) и сверхбольшим (СБИС, VLSI)
интегральным микросхемам.

Особенно сильное влияние на развитие цифровых вычислительных
устройств оказало создание первых микропроцессоров, что привело
к широкому внедрению цифровых технологий обработки информации.
Однако микропроцессоры не всегда приемлимы при решении задач в
цифровой схемотехники: работа микропроцессора основана на микропрограмме
и состоит из последовательности шагов конечной длительности, в то время
как для многих задач (в том числе, связанных и с обеспечением работы
самих МП) требуются устройства с минимальной задержкой выполнения
логических функций, что может быть обеспечено одним из трёх способов:

\begin{itemize}
\item Использованием наборов стандартной дискретной цифровой логи-
ки общего применения, например, наборов логических микросхем
74-й или 4000-й серии (TTL, CMOS) и типовых периферийных БИС из микропроцессорного комплекта;
\item Использованием заказных ИС (так называемых ASIC, \engl{appication-specific integrated circuit});
\item Использованием программируемых логических интегральных схем (PLD - Programmable Logic Devices).
\end{itemize}

\pagebreak
}%			!	для заполнения			!
%----8<-------cut here-------------8<-------------------------8<----------------
% Включайте сюда своё содержимое.
% Если какие-то из разделов работы не нужны, их можно закомментировать.
{
\pagestyle{framedcontent}
\cwchapter{Раздел 1}
Смотрите в файлик \texttt{20\_main\_part\_1.tex} для примеров добавления картинок,
списков, и всего подобного.

%-------------------------------------------------------------------------------
% Пример определения картинки слева от текста
%-------------------------------------------------------------------------------
%\begin{wrapfigure}[12]{r}{0.34\linewidth}
%\includegraphics[scale=0.35]{images/7400.jpg}
%\caption{Микросхема дискретной логики 4И (7400).}
%\label{ic-7400}
%\end{wrapfigure}

%-------------------------------------------------------------------------------
% Пример определения точечного списка по левому краю
%-------------------------------------------------------------------------------
%\begin{flushleft}
%\begin{itemize}
%\item Базовые логические элементы (NOT, AND, OR);
%\item Более сложные логические элементы (полусумматор, XOR);
%\item Память (триггеры, регистры и их наборы, RAM, PROM);
%\item Счетчики, компараторы;
%\item Мультиплексоры, шифраторы, дешифраторы;
%\end{itemize}
%\end{flushleft}

%-------------------------------------------------------------------------------
% Пример определения картинки сначала листа
%-------------------------------------------------------------------------------
%\begin{figure}[htp]
%\centering
%\includegraphics[scale=0.30]{images/TRS80M100_inside.jpg}
%\caption{Материнская плата Model 100.}
%\label{trs80-mobo}
%\end{figure}

%-------------------------------------------------------------------------------
% Пример определения кода Verilog
%-------------------------------------------------------------------------------
%\begin{minipage}{\linewidth}
%Пример кода Verilog:
%\lstinputlisting[language=Verilog]{code_examples/verilog_count.v}
%\end{minipage}

\cwsection{Секция 1}
lorem ipsum
\cwsection{Секция 2}
ipsum lorem
\cwsection{Секция 3}
to be or not to be, that is the question

\newpage
}
%
%   ....
%
%----8<-------cut here-------------8<-------------------------8<----------------
%
% Заключение
%
{
\pagestyle{framedcontent}
\cwNchapter{Заключение}
To be filled...

\vfill
\newpage
}%	!	для заполнения			!
%
% Список использованных источников
%
% use \addBook{author}{name}{city}{publisher}{year}{pages}
% use \addURL{site title}{site organization}{year}{URL}
{
\pagestyle{framedcontent}
\cwNchapter{Список использованных источников}
\begin{enumerate}
%--------------------
\item \addBook
{Антонов А.П.}
{Язык описания цифровых устройств. ALTERA HDL. Практический курс.}
{М.}
{ИП Радио Софт}{2002}{224}
%--------------------
\item \addBook
{Грушвицкий Р.И., Мурсаев А.Х., Угрюмов Е.П.}
{Проектирование систем на микросхемах программируемой логики.}
{СПб.}
{БХВ-Петербург}{2002}{608}
%--------------------
\item \addBook
{Зобенко А. А., Филиппов А. С., Комолов Д. А., Мяльк Р. А.}
{Системы автоматизированного проектирования фирмы Altera MAX+plus II и Quartus II. Краткое описание и самоучитель.}
{М.}
{ИП Радио Софт}{2002}{360}
%--------------------
\item \addBook
{Стешенко В.Б.}
{ПЛИС фирмы “ALTERA”: элементая база, система проектирования и языки описания аппаратуры.}
{М.}
{Издательский дом, ДОДЕКА -- XXI}{2002}{567}
%--------------------
\item \addURL
{Википедия свободная энциклопедия}
{Wikimedia Foundation, Inc}
{2013}
{http://ru.wikipedia.org/wiki/ПЛИС}
%--------------------
\item \addURL
{FPGA Architecture for the Challenge}
{The Edward S. Rogers Sr. Department of Electrical and Computer Engineering}
{1998}
{http://www.eecg.toronto.edu/\~{}vaughn/challenge/ fpga\_arch.html}
%--------------------
\label{literarure:num}
\end{enumerate}
\vfill
\newpage
}%	!	для заполнения			!
}
\end{document}


