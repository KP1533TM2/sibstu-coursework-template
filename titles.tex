%
%  Форматирование заголовков
%

%\newcommand{\addURL}[4]{{#1} [Электронный ресурс]: электронная система: база данных содержит электронные статьи / {#2}, {#3}. -- Режим доступа: {#4}}

\newcommand{\addURL}[4]{Электронный ресурс {#4}}

\newcommand{\addBook}[6]{{#1} {#2} -- {#3}: {#4}, {#5}. -- {#6} c.}

% ACHTUNG, ACHTUNG! Юзаем \cwchapter и \cwsection для правильного
% оформления и автоматических ништяков :-)

\newcommand\cwchapter[1]{\chapter{\MakeUppercase{#1}}}
\newcommand\cwNchapter[1]{\chapter*{\MakeUppercase{#1}}\addcontentsline{toc}{chapter}{\MakeUppercase{#1}}}
\newcommand\cwNNchapter[1]{\chapter*{\MakeUppercase{#1}}}
\newcommand\cwsection[1]{\section{\MakeUppercase{#1}}}
\newcommand\cwNsection[1]{\section*{\MakeUppercase{#1}}\addcontentsline{toc}{section}{\MakeUppercase{#1}}}

% Эти строки определяют вид заголовков разделов и секций
% MakeUppercase здесь почему-то не работает (не разбирался), так что
% обойдемся костылями B-)
\titleformat{\chapter}[hang]{\fontsize{16pt}{13pt}\selectfont}{\thechapter\quad}{-6pt}{}
\titleformat{\section}[hang]{\fontsize{16pt}{13pt}\selectfont}{\thesection\quad}{-6pt}{}
\titleclass{\chapter}{straight}
\titleclass{\section}{straight}

% Отступы от заголовков
\titlespacing{\chapter}{0mm}{1.2em}{1.2em}
\titlespacing{\section}{0mm}{1.2em}{1.2em}

% Отступ красной строки
\setlength{\parindent}{1cm}

% Промежутки между параграфами
\setlength{\parskip}{.5\baselineskip}

% Эти строки описывают подписи к картинкам 
\usepackage{caption}
\captionsetup[figure]{name=Рисунок, justification=centering,labelsep=endash}